\documentclass[]{article}
\usepackage{lmodern}
\usepackage{amssymb,amsmath}
\usepackage{ifxetex,ifluatex}
\usepackage{fixltx2e} % provides \textsubscript
\ifnum 0\ifxetex 1\fi\ifluatex 1\fi=0 % if pdftex
  \usepackage[T1]{fontenc}
  \usepackage[utf8]{inputenc}
\else % if luatex or xelatex
  \ifxetex
    \usepackage{mathspec}
  \else
    \usepackage{fontspec}
  \fi
  \defaultfontfeatures{Ligatures=TeX,Scale=MatchLowercase}
\fi
% use upquote if available, for straight quotes in verbatim environments
\IfFileExists{upquote.sty}{\usepackage{upquote}}{}
% use microtype if available
\IfFileExists{microtype.sty}{%
\usepackage{microtype}
\UseMicrotypeSet[protrusion]{basicmath} % disable protrusion for tt fonts
}{}
\usepackage[margin=1in]{geometry}
\usepackage{hyperref}
\hypersetup{unicode=true,
            pdftitle={Ciclo de vida de los datos - P2},
            pdfauthor={Jorge Alaiza},
            pdfborder={0 0 0},
            breaklinks=true}
\urlstyle{same}  % don't use monospace font for urls
\usepackage{color}
\usepackage{fancyvrb}
\newcommand{\VerbBar}{|}
\newcommand{\VERB}{\Verb[commandchars=\\\{\}]}
\DefineVerbatimEnvironment{Highlighting}{Verbatim}{commandchars=\\\{\}}
% Add ',fontsize=\small' for more characters per line
\usepackage{framed}
\definecolor{shadecolor}{RGB}{248,248,248}
\newenvironment{Shaded}{\begin{snugshade}}{\end{snugshade}}
\newcommand{\KeywordTok}[1]{\textcolor[rgb]{0.13,0.29,0.53}{\textbf{#1}}}
\newcommand{\DataTypeTok}[1]{\textcolor[rgb]{0.13,0.29,0.53}{#1}}
\newcommand{\DecValTok}[1]{\textcolor[rgb]{0.00,0.00,0.81}{#1}}
\newcommand{\BaseNTok}[1]{\textcolor[rgb]{0.00,0.00,0.81}{#1}}
\newcommand{\FloatTok}[1]{\textcolor[rgb]{0.00,0.00,0.81}{#1}}
\newcommand{\ConstantTok}[1]{\textcolor[rgb]{0.00,0.00,0.00}{#1}}
\newcommand{\CharTok}[1]{\textcolor[rgb]{0.31,0.60,0.02}{#1}}
\newcommand{\SpecialCharTok}[1]{\textcolor[rgb]{0.00,0.00,0.00}{#1}}
\newcommand{\StringTok}[1]{\textcolor[rgb]{0.31,0.60,0.02}{#1}}
\newcommand{\VerbatimStringTok}[1]{\textcolor[rgb]{0.31,0.60,0.02}{#1}}
\newcommand{\SpecialStringTok}[1]{\textcolor[rgb]{0.31,0.60,0.02}{#1}}
\newcommand{\ImportTok}[1]{#1}
\newcommand{\CommentTok}[1]{\textcolor[rgb]{0.56,0.35,0.01}{\textit{#1}}}
\newcommand{\DocumentationTok}[1]{\textcolor[rgb]{0.56,0.35,0.01}{\textbf{\textit{#1}}}}
\newcommand{\AnnotationTok}[1]{\textcolor[rgb]{0.56,0.35,0.01}{\textbf{\textit{#1}}}}
\newcommand{\CommentVarTok}[1]{\textcolor[rgb]{0.56,0.35,0.01}{\textbf{\textit{#1}}}}
\newcommand{\OtherTok}[1]{\textcolor[rgb]{0.56,0.35,0.01}{#1}}
\newcommand{\FunctionTok}[1]{\textcolor[rgb]{0.00,0.00,0.00}{#1}}
\newcommand{\VariableTok}[1]{\textcolor[rgb]{0.00,0.00,0.00}{#1}}
\newcommand{\ControlFlowTok}[1]{\textcolor[rgb]{0.13,0.29,0.53}{\textbf{#1}}}
\newcommand{\OperatorTok}[1]{\textcolor[rgb]{0.81,0.36,0.00}{\textbf{#1}}}
\newcommand{\BuiltInTok}[1]{#1}
\newcommand{\ExtensionTok}[1]{#1}
\newcommand{\PreprocessorTok}[1]{\textcolor[rgb]{0.56,0.35,0.01}{\textit{#1}}}
\newcommand{\AttributeTok}[1]{\textcolor[rgb]{0.77,0.63,0.00}{#1}}
\newcommand{\RegionMarkerTok}[1]{#1}
\newcommand{\InformationTok}[1]{\textcolor[rgb]{0.56,0.35,0.01}{\textbf{\textit{#1}}}}
\newcommand{\WarningTok}[1]{\textcolor[rgb]{0.56,0.35,0.01}{\textbf{\textit{#1}}}}
\newcommand{\AlertTok}[1]{\textcolor[rgb]{0.94,0.16,0.16}{#1}}
\newcommand{\ErrorTok}[1]{\textcolor[rgb]{0.64,0.00,0.00}{\textbf{#1}}}
\newcommand{\NormalTok}[1]{#1}
\usepackage{graphicx,grffile}
\makeatletter
\def\maxwidth{\ifdim\Gin@nat@width>\linewidth\linewidth\else\Gin@nat@width\fi}
\def\maxheight{\ifdim\Gin@nat@height>\textheight\textheight\else\Gin@nat@height\fi}
\makeatother
% Scale images if necessary, so that they will not overflow the page
% margins by default, and it is still possible to overwrite the defaults
% using explicit options in \includegraphics[width, height, ...]{}
\setkeys{Gin}{width=\maxwidth,height=\maxheight,keepaspectratio}
\IfFileExists{parskip.sty}{%
\usepackage{parskip}
}{% else
\setlength{\parindent}{0pt}
\setlength{\parskip}{6pt plus 2pt minus 1pt}
}
\setlength{\emergencystretch}{3em}  % prevent overfull lines
\providecommand{\tightlist}{%
  \setlength{\itemsep}{0pt}\setlength{\parskip}{0pt}}
\setcounter{secnumdepth}{0}
% Redefines (sub)paragraphs to behave more like sections
\ifx\paragraph\undefined\else
\let\oldparagraph\paragraph
\renewcommand{\paragraph}[1]{\oldparagraph{#1}\mbox{}}
\fi
\ifx\subparagraph\undefined\else
\let\oldsubparagraph\subparagraph
\renewcommand{\subparagraph}[1]{\oldsubparagraph{#1}\mbox{}}
\fi

%%% Use protect on footnotes to avoid problems with footnotes in titles
\let\rmarkdownfootnote\footnote%
\def\footnote{\protect\rmarkdownfootnote}

%%% Change title format to be more compact
\usepackage{titling}

% Create subtitle command for use in maketitle
\newcommand{\subtitle}[1]{
  \posttitle{
    \begin{center}\large#1\end{center}
    }
}

\setlength{\droptitle}{-2em}

  \title{Ciclo de vida de los datos - P2}
    \pretitle{\vspace{\droptitle}\centering\huge}
  \posttitle{\par}
    \author{Jorge Alaiza}
    \preauthor{\centering\large\emph}
  \postauthor{\par}
      \predate{\centering\large\emph}
  \postdate{\par}
    \date{January 2, 2019}


\begin{document}
\maketitle

\begin{center}\rule{0.5\linewidth}{\linethickness}\end{center}

\section{Ciclo de vida de los datos.}\label{ciclo-de-vida-de-los-datos.}

\subsubsection{\texorpdfstring{1. \emph{Descripción del dataset. ¿Por
qué es importante y qué pregunta/problema pretende
responder?}}{1. Descripción del dataset. ¿Por qué es importante y qué pregunta/problema pretende responder?}}\label{descripcion-del-dataset.-por-que-es-importante-y-que-preguntaproblema-pretende-responder}

\begin{verbatim}
Para esta práctica he decidido realizarla sobre uno de los datasets de Kaggle, este dataset esta orientado a la recolección de datos sobre registros de suicidios, estos datos por razones obvias vienen enmascarados para proteger la identidad de los afectados por lo que no aparece información certera de la edad del sujeto u otra información que pueda afectar a su privacidad.

este conjunto de datos contiene la siguiente información:

Country: Pais donde ocurre el suicidio
Year: año en el que este suicidio ocurre
Sex: Sexo del conjunto de personas que lo han cometido
Age: Rango de edad de las personas que se suicidan
suicides_no: Volumen de suicidios cometidos por pais/anyo/sexo y rango de edad
population: Volumen de población que forma el subconjunto por pais/anyo/sexo/rango de edad

El valor que realmente se quiere extraer sobre este dataaset es el relativo a que conjuntos sociales les afecta mas la necesidad de suicidarse, por pais, edad o sexo (a grandes rasgos, no se tiene información sobre el trabajo u otros componentes que pudieran afectar a que finalmente se suicidara)
\end{verbatim}

\subsubsection{\texorpdfstring{2. \emph{Integración y selección de los
datos de interés a
analizar.}}{2. Integración y selección de los datos de interés a analizar.}}\label{integracion-y-seleccion-de-los-datos-de-interes-a-analizar.}

\begin{Shaded}
\begin{Highlighting}[]
\KeywordTok{library}\NormalTok{(readr)}

\NormalTok{suicide_raw_data<-}\StringTok{ }\KeywordTok{read_delim}\NormalTok{(}\StringTok{"/home/alaiza/Desktop/who-suicide-statistics/who_suicide_statistics.csv"}\NormalTok{, }
                       \StringTok{","}\NormalTok{, }\DataTypeTok{escape_double =} \OtherTok{FALSE}\NormalTok{,}
                       \DataTypeTok{trim_ws =} \OtherTok{TRUE}\NormalTok{)}
\end{Highlighting}
\end{Shaded}

\begin{verbatim}
## Parsed with column specification:
## cols(
##   country = col_character(),
##   year = col_double(),
##   sex = col_character(),
##   age = col_character(),
##   suicides_no = col_double(),
##   population = col_double()
## )
\end{verbatim}

Para este estudio, al solo disponer de unas pocas columnas que considero
de gran valor no se van a filtrar en primera instancia, se van a
utilizar las 6 columnas.

\subsubsection{\texorpdfstring{3. \emph{Limpieza de los
datos}}{3. Limpieza de los datos}}\label{limpieza-de-los-datos}

\paragraph{\texorpdfstring{3.1 \emph{¿Los datos contienen ceros o
elementos vacíos? ¿Cómo gestionarías cada uno de estos
casos?.}}{3.1 ¿Los datos contienen ceros o elementos vacíos? ¿Cómo gestionarías cada uno de estos casos?.}}\label{los-datos-contienen-ceros-o-elementos-vacios-como-gestionarias-cada-uno-de-estos-casos.}

\begin{Shaded}
\begin{Highlighting}[]
\NormalTok{missingvalues <-}\StringTok{ }\ControlFlowTok{function}\NormalTok{(array)\{}
  \ControlFlowTok{for}\NormalTok{ (j }\ControlFlowTok{in} \DecValTok{1}\OperatorTok{:}\KeywordTok{length}\NormalTok{(array)) \{}
    \ControlFlowTok{if}\NormalTok{(}\KeywordTok{is.na}\NormalTok{(array[j]))\{}
      \KeywordTok{return}\NormalTok{(}\StringTok{'yes'}\NormalTok{)}
\NormalTok{    \}}
\NormalTok{  \}}
  \KeywordTok{return}\NormalTok{(}\StringTok{'no'}\NormalTok{)}
\NormalTok{\}}

\KeywordTok{missingvalues}\NormalTok{(suicide_raw_data}\OperatorTok{$}\NormalTok{country)}
\end{Highlighting}
\end{Shaded}

\begin{verbatim}
## [1] "no"
\end{verbatim}

\begin{Shaded}
\begin{Highlighting}[]
\KeywordTok{missingvalues}\NormalTok{(suicide_raw_data}\OperatorTok{$}\NormalTok{year)}
\end{Highlighting}
\end{Shaded}

\begin{verbatim}
## [1] "no"
\end{verbatim}

\begin{Shaded}
\begin{Highlighting}[]
\KeywordTok{missingvalues}\NormalTok{(suicide_raw_data}\OperatorTok{$}\NormalTok{sex)}
\end{Highlighting}
\end{Shaded}

\begin{verbatim}
## [1] "no"
\end{verbatim}

\begin{Shaded}
\begin{Highlighting}[]
\KeywordTok{missingvalues}\NormalTok{(suicide_raw_data}\OperatorTok{$}\NormalTok{age)}
\end{Highlighting}
\end{Shaded}

\begin{verbatim}
## [1] "no"
\end{verbatim}

\begin{Shaded}
\begin{Highlighting}[]
\KeywordTok{missingvalues}\NormalTok{(suicide_raw_data}\OperatorTok{$}\NormalTok{suicides_no)}
\end{Highlighting}
\end{Shaded}

\begin{verbatim}
## [1] "yes"
\end{verbatim}

\begin{Shaded}
\begin{Highlighting}[]
\KeywordTok{missingvalues}\NormalTok{(suicide_raw_data}\OperatorTok{$}\NormalTok{population)}
\end{Highlighting}
\end{Shaded}

\begin{verbatim}
## [1] "yes"
\end{verbatim}

Se puede observar que para el número de suicidios y volumen de la
población hay valores nulos, en este caso como solo queremos la
información que nos indique la tasa de suicidios que están debidamente
documentados se procederá a eliminar las filas con estos valores, por
otro lado el que existan valores definidos a 0 (que son muchos los
casos) indicarán que no hay suicidios (si es que los datos estan bien
construidos), pero hay otros controles que debemos hacer previamente
sobre los datos:

\begin{Shaded}
\begin{Highlighting}[]
\ControlFlowTok{for}\NormalTok{ (j }\ControlFlowTok{in} \DecValTok{1}\OperatorTok{:}\KeywordTok{length}\NormalTok{(suicide_raw_data}\OperatorTok{$}\NormalTok{suicides_no)) \{}
  \ControlFlowTok{if}\NormalTok{((}\OperatorTok{!}\KeywordTok{is.na}\NormalTok{(suicide_raw_data}\OperatorTok{$}\NormalTok{suicides_no[j]) }\OperatorTok{&}\StringTok{ }\OperatorTok{!}\KeywordTok{is.na}\NormalTok{(suicide_raw_data}\OperatorTok{$}\NormalTok{population[j])))\{}
    \ControlFlowTok{if}\NormalTok{(suicide_raw_data}\OperatorTok{$}\NormalTok{suicides_no[j]}\OperatorTok{>=}\NormalTok{suicide_raw_data}\OperatorTok{$}\NormalTok{population[j])\{}
      \KeywordTok{print}\NormalTok{(}\StringTok{'Hay valores sin lógica')}
\StringTok{    \}}
\StringTok{    if(suicide_raw_data$population[j]==0)\{}
\StringTok{      print('}\NormalTok{Hay valores sin lógica')}
\NormalTok{    \}}
\NormalTok{  \}}
\NormalTok{\}}
\end{Highlighting}
\end{Shaded}

\begin{verbatim}
Con el código anterior, al no imprimir nada se sabe que dentro de las filas que nos van a quedar no hay incongruencias del tipo que haya mas suicidios que población o haya poblacion igual a 0 (algo posible pero muy extraño)
\end{verbatim}

\begin{Shaded}
\begin{Highlighting}[]
\NormalTok{data_fixed <-}\StringTok{ }\NormalTok{suicide_raw_data}
\NormalTok{data_fixed <-}\StringTok{ }\NormalTok{data_fixed[}\KeywordTok{complete.cases}\NormalTok{(data_fixed), ]}
\end{Highlighting}
\end{Shaded}

\paragraph{\texorpdfstring{3.2 \emph{Identificación y tratamiento de
valores
extremos.}}{3.2 Identificación y tratamiento de valores extremos.}}\label{identificacion-y-tratamiento-de-valores-extremos.}

Los valores extremos que se van a buscar al estar asociados a un
conjunto de población (por sexo y edad) lo mejor va a ser calcular los
valores extremos del ratio entre suicidios y poblacion perteneciente a
ese subconjunto de poblacion, o lo que es lo mismo, la division entre
``suicides\_no'' y ``population''

\begin{Shaded}
\begin{Highlighting}[]
\NormalTok{ratio <-}\StringTok{ }\NormalTok{data_fixed}\OperatorTok{$}\NormalTok{suicides_no}

\ControlFlowTok{for}\NormalTok{ (j }\ControlFlowTok{in} \DecValTok{1}\OperatorTok{:}\KeywordTok{length}\NormalTok{(ratio)) \{}
  \ControlFlowTok{if}\NormalTok{(ratio[j]}\OperatorTok{!=}\DecValTok{0}\NormalTok{)\{}
\NormalTok{    ratio[j]<-}\StringTok{ }\NormalTok{data_fixed}\OperatorTok{$}\NormalTok{suicides_no[j]}\OperatorTok{/}\NormalTok{data_fixed}\OperatorTok{$}\NormalTok{population[j]}\OperatorTok{*}\DecValTok{1000}
\NormalTok{  \}}
\NormalTok{\}}

\NormalTok{data_fixed <-}\StringTok{ }\KeywordTok{invisible}\NormalTok{(}\KeywordTok{cbind}\NormalTok{(data_fixed, ratio))}

\KeywordTok{head}\NormalTok{(data_fixed, }\DataTypeTok{n=}\DecValTok{10}\NormalTok{)}
\end{Highlighting}
\end{Shaded}

\begin{verbatim}
##    country year    sex         age suicides_no population      ratio
## 1  Albania 1987 female 15-24 years          14     289700 0.04832585
## 2  Albania 1987 female 25-34 years           4     257200 0.01555210
## 3  Albania 1987 female 35-54 years           6     278800 0.02152080
## 4  Albania 1987 female  5-14 years           0     311000 0.00000000
## 5  Albania 1987 female 55-74 years           0     144600 0.00000000
## 6  Albania 1987 female   75+ years           1      35600 0.02808989
## 7  Albania 1987   male 15-24 years          21     312900 0.06711409
## 8  Albania 1987   male 25-34 years           9     274300 0.03281079
## 9  Albania 1987   male 35-54 years          16     308000 0.05194805
## 10 Albania 1987   male  5-14 years           0     338200 0.00000000
\end{verbatim}

\begin{Shaded}
\begin{Highlighting}[]
\KeywordTok{summary}\NormalTok{(data_fixed}\OperatorTok{$}\NormalTok{ratio)}
\end{Highlighting}
\end{Shaded}

\begin{verbatim}
##     Min.  1st Qu.   Median     Mean  3rd Qu.     Max. 
## 0.000000 0.007818 0.059756 0.131851 0.170101 3.007519
\end{verbatim}

\begin{Shaded}
\begin{Highlighting}[]
\KeywordTok{boxplot}\NormalTok{(data_fixed}\OperatorTok{$}\NormalTok{ratio)}
\end{Highlighting}
\end{Shaded}

\includegraphics{Ciclo_vida_datos_P2_files/figure-latex/unnamed-chunk-5-1.pdf}

se puede observar que hay muchos valores considerables ``atípicos'' o
por lo menos que merecen ser revisados

\begin{Shaded}
\begin{Highlighting}[]
\NormalTok{vector2 <-}\StringTok{  }\NormalTok{data_fixed}\OperatorTok{$}\NormalTok{ratio[ data_fixed}\OperatorTok{$}\NormalTok{ratio }\OperatorTok{!=}\StringTok{ }\DecValTok{0}\NormalTok{ ] }
\KeywordTok{boxplot}\NormalTok{(vector2)}
\end{Highlighting}
\end{Shaded}

\includegraphics{Ciclo_vida_datos_P2_files/figure-latex/unnamed-chunk-6-1.pdf}

se puede ver que el ``ruido'' que pueden meter los paises pequeños y sin
casos de suicidio no cambia notablemente el diagrama, por alguna razón
son anormales estos valores, se procede a averiguar mas información de
esos valores.

\begin{Shaded}
\begin{Highlighting}[]
\NormalTok{country <-}\StringTok{  }\NormalTok{data_fixed}\OperatorTok{$}\NormalTok{country [data_fixed}\OperatorTok{$}\NormalTok{ratio}\OperatorTok{>=}\StringTok{ }\FloatTok{1.5}\NormalTok{ ] }
\NormalTok{year <-}\StringTok{  }\NormalTok{data_fixed}\OperatorTok{$}\NormalTok{year [data_fixed}\OperatorTok{$}\NormalTok{ratio}\OperatorTok{>=}\StringTok{ }\FloatTok{1.5}\NormalTok{ ] }
\NormalTok{sex <-}\StringTok{  }\NormalTok{data_fixed}\OperatorTok{$}\NormalTok{sex [data_fixed}\OperatorTok{$}\NormalTok{ratio}\OperatorTok{>=}\StringTok{ }\FloatTok{1.5}\NormalTok{ ] }
\NormalTok{age <-}\StringTok{  }\NormalTok{data_fixed}\OperatorTok{$}\NormalTok{age [data_fixed}\OperatorTok{$}\NormalTok{ratio}\OperatorTok{>=}\StringTok{ }\FloatTok{1.5}\NormalTok{ ]}
\NormalTok{ratio <-}\StringTok{  }\NormalTok{data_fixed}\OperatorTok{$}\NormalTok{ratio [data_fixed}\OperatorTok{$}\NormalTok{ratio}\OperatorTok{>=}\StringTok{ }\FloatTok{1.5}\NormalTok{ ]}

\NormalTok{dataframe_atipicos <-}\StringTok{ }\KeywordTok{invisible}\NormalTok{(}\KeywordTok{cbind}\NormalTok{(country, year,sex,age,ratio))}
\NormalTok{extremecases <-}\StringTok{ }\NormalTok{dataframe_atipicos[}\KeywordTok{order}\NormalTok{(ratio),]}

\KeywordTok{tail}\NormalTok{(extremecases, }\DataTypeTok{n=}\DecValTok{10}\NormalTok{)}
\end{Highlighting}
\end{Shaded}

\begin{verbatim}
##       country         year   sex    age         ratio             
## [33,] "Hungary"       "1990" "male" "75+ years" "1.96441657845452"
## [34,] "Hungary"       "1980" "male" "75+ years" "2.02151755379388"
## [35,] "Seychelles"    "2006" "male" "75+ years" "2.04918032786885"
## [36,] "Hungary"       "1985" "male" "75+ years" "2.07169214549288"
## [37,] "Hungary"       "1982" "male" "75+ years" "2.09650582362729"
## [38,] "Hungary"       "1979" "male" "75+ years" "2.11136890951276"
## [39,] "Hungary"       "1981" "male" "75+ years" "2.19224283305228"
## [40,] "Aruba"         "1995" "male" "75+ years" "2.24971878515186"
## [41,] "French Guiana" "1979" "male" "75+ years" "2.5"             
## [42,] "San Marino"    "1997" "male" "75+ years" "3.00751879699248"
\end{verbatim}

por lo que parece, los casos mas extremos son gente de +75 años, que son
un numero relevante de casos sobre una poblacion muy pequeña (la de
mayores de 75 años) lo cual me hace pensar que se estan considerando los
casos de eutanasia como suicidio, para poder hacer un estudio algo mas
interesante (y de alguna forma poder disipar del estudio casos
voluntarios de suicidio) para tratar casos de suicidio voluntario no
relacionados con muertes naturales se van a eliminar el subconjunto de
personas mayores a 75 años

**Aclaracion: he investigado los paises que aparecian y sin pararme en
todos, parece que permiten la eutanasia en distintas formas y con
distintas regulaciones.

\begin{Shaded}
\begin{Highlighting}[]
\NormalTok{data_fixed_v2<-}\KeywordTok{subset}\NormalTok{(data_fixed, data_fixed}\OperatorTok{$}\NormalTok{age}\OperatorTok{!=}\StringTok{"75+ years"}\NormalTok{)}
\KeywordTok{boxplot}\NormalTok{(data_fixed_v2}\OperatorTok{$}\NormalTok{ratio)}
\end{Highlighting}
\end{Shaded}

\includegraphics{Ciclo_vida_datos_P2_files/figure-latex/unnamed-chunk-8-1.pdf}

Se puede observar una mejoria notable en los resultados, sin embargo
estos valores tan por encima es posible que respondan a una realidad que
debamos investigar, conocimiento que debamos de realizar tras un
analisis mas profundo.

\subsubsection{\texorpdfstring{4. \emph{Análisis de los
datos.}}{4. Análisis de los datos.}}\label{analisis-de-los-datos.}

\paragraph{\texorpdfstring{4.1 \emph{Selección de los grupos de datos
que se quieren analizar/comparar (planificación de los análisis a
aplicar).}}{4.1 Selección de los grupos de datos que se quieren analizar/comparar (planificación de los análisis a aplicar).}}\label{seleccion-de-los-grupos-de-datos-que-se-quieren-analizarcomparar-planificacion-de-los-analisis-a-aplicar.}

\textbf{\emph{PUNTO 1}} En este analisis se van a comprobar las
diferencias entre hombres y mujeres comprobando que la media de los
hombres es claramente superior a la de las mujeres con un \% de
confianza del 97\% (un nivel bastante alto de confianza)

\textbf{\emph{PUNTO 2}} Un segundo punto a analizar sera comprobar si
existen diferencias entre paises con gran volumen de poblacion y los de
menor poblacion, dividiendo estos conjuntos por la mitad (divididos por
la mediana, para dividir al 50\% los volumenes de poblaciones y sus
paises asociados).

\subsubsection{\texorpdfstring{\textbf{\emph{PUNTO
1}}}{PUNTO 1}}\label{punto-1}

primeramente a nivel exploratorio, se comprueba si para exactamente el
mismo numero de paises y rangos de edad existen diferencias en el total
de suicidios

\begin{Shaded}
\begin{Highlighting}[]
\NormalTok{data_male<-data_fixed_v2[data_fixed_v2}\OperatorTok{$}\NormalTok{sex}\OperatorTok{==}\StringTok{"male"}\NormalTok{, ]}
\NormalTok{data_female<-data_fixed_v2[data_fixed_v2}\OperatorTok{$}\NormalTok{sex}\OperatorTok{==}\StringTok{"female"}\NormalTok{, ]}


\NormalTok{slices <-}\StringTok{ }\KeywordTok{c}\NormalTok{(}\KeywordTok{sum}\NormalTok{(data_male}\OperatorTok{$}\NormalTok{suicides_no), }\KeywordTok{sum}\NormalTok{(data_female}\OperatorTok{$}\NormalTok{suicides_no))}
\NormalTok{lbls <-}\StringTok{ }\KeywordTok{c}\NormalTok{(}\StringTok{"male total"}\NormalTok{, }\StringTok{"female total"}\NormalTok{)}
\NormalTok{pct <-}\StringTok{ }\KeywordTok{round}\NormalTok{(slices}\OperatorTok{/}\KeywordTok{sum}\NormalTok{(slices)}\OperatorTok{*}\DecValTok{100}\NormalTok{)}
\NormalTok{lbls <-}\StringTok{ }\KeywordTok{paste}\NormalTok{(lbls, pct) }\CommentTok{# add percents to labels }
\NormalTok{lbls <-}\StringTok{ }\KeywordTok{paste}\NormalTok{(lbls,}\StringTok{"%"}\NormalTok{,}\DataTypeTok{sep=}\StringTok{""}\NormalTok{) }\CommentTok{# ad % to labels }
\KeywordTok{pie}\NormalTok{(slices,}\DataTypeTok{labels =}\NormalTok{ lbls, }\DataTypeTok{col=}\KeywordTok{rainbow}\NormalTok{(}\KeywordTok{length}\NormalTok{(lbls)),}
    \DataTypeTok{main=}\StringTok{"Pie Chart comparisson number of suicides per sex"}\NormalTok{) }
\end{Highlighting}
\end{Shaded}

\includegraphics{Ciclo_vida_datos_P2_files/figure-latex/unnamed-chunk-9-1.pdf}

para comenzar, como primera aproximación, es bastante característico el
que el volumen de hombres que se suicidan ocupe el 77\%, a continuación,
es esperable que el volumen de poblacion entre hombres y mujeres sea el
mismo, por lo que se va a comprobar.

\begin{Shaded}
\begin{Highlighting}[]
\NormalTok{slices <-}\StringTok{ }\KeywordTok{c}\NormalTok{(}\KeywordTok{sum}\NormalTok{(data_male}\OperatorTok{$}\NormalTok{population), }\KeywordTok{sum}\NormalTok{(data_female}\OperatorTok{$}\NormalTok{population))}
\NormalTok{lbls <-}\StringTok{ }\KeywordTok{c}\NormalTok{(}\StringTok{"male total"}\NormalTok{, }\StringTok{"female total"}\NormalTok{)}
\NormalTok{pct <-}\StringTok{ }\KeywordTok{round}\NormalTok{(slices}\OperatorTok{/}\KeywordTok{sum}\NormalTok{(slices)}\OperatorTok{*}\DecValTok{100}\NormalTok{)}
\NormalTok{lbls <-}\StringTok{ }\KeywordTok{paste}\NormalTok{(lbls, pct) }\CommentTok{# add percents to labels }
\NormalTok{lbls <-}\StringTok{ }\KeywordTok{paste}\NormalTok{(lbls,}\StringTok{"%"}\NormalTok{,}\DataTypeTok{sep=}\StringTok{""}\NormalTok{) }\CommentTok{# ad % to labels }
\KeywordTok{pie}\NormalTok{(slices,}\DataTypeTok{labels =}\NormalTok{ lbls, }\DataTypeTok{col=}\KeywordTok{rainbow}\NormalTok{(}\KeywordTok{length}\NormalTok{(lbls)),}
    \DataTypeTok{main=}\StringTok{"Pie Chart comparisson number of suicides per sex"}\NormalTok{)}
\end{Highlighting}
\end{Shaded}

\includegraphics{Ciclo_vida_datos_P2_files/figure-latex/unnamed-chunk-10-1.pdf}

Esta claro que los resultados en primera instancia muestran
clarisimamente que los hombres se suicidan mas de un 250\% mas que las
mujeres, pero hay que comprobar que este resultado extraido visualmente
se puede asegurar con un nivel de significancia alto, como se ha pedido
anteriormente, del 97\%.

Para este estudio se va a utilizar el ratio, ya que comprende la
division entre el numero de suicidios y el volumen de poblacion.

Recapitulando, se va a comprobar si : \textbf{\emph{H0: las mujeres
tienen la misma tasa de suicidios inferior a los hombres, por el
contrario, H1: las mujeres tienen una tasa igual o superior.}}

\begin{Shaded}
\begin{Highlighting}[]
\CommentTok{#97% ---> alfa=1-0.97 --> alfa= 0.03}
\CommentTok{#P(Z<z) = 1-alfa/2 = 1-0.03/2 = 0.985}
\CommentTok{#Según las tablas P(Z<z)=0.985 --> z=2.17}
\NormalTok{var1 <-}\StringTok{ }\NormalTok{data_male}\OperatorTok{$}\NormalTok{ratio}
\NormalTok{##por un lado:}
\NormalTok{extremosuperior <-}\StringTok{ }\KeywordTok{mean}\NormalTok{(var1) }\OperatorTok{+}\StringTok{ }\FloatTok{2.17} \OperatorTok{*}\StringTok{ }\KeywordTok{sd}\NormalTok{(var1)}\OperatorTok{/}\KeywordTok{sqrt}\NormalTok{(}\KeywordTok{length}\NormalTok{(var1))}
\NormalTok{extremoinferior <-}\StringTok{ }\KeywordTok{mean}\NormalTok{(var1) }\OperatorTok{-}\StringTok{ }\FloatTok{2.17} \OperatorTok{*}\StringTok{ }\KeywordTok{sd}\NormalTok{(var1)}\OperatorTok{/}\KeywordTok{sqrt}\NormalTok{(}\KeywordTok{length}\NormalTok{(var1))}
\KeywordTok{cat}\NormalTok{(}\StringTok{"rango de aceptacion: ["}\NormalTok{, extremoinferior,}\StringTok{","}\NormalTok{,extremosuperior,}\StringTok{"]"}\NormalTok{)}
\end{Highlighting}
\end{Shaded}

\begin{verbatim}
## rango de aceptacion: [ 0.1658331 , 0.1726431 ]
\end{verbatim}

por lo que la media de ratio de las mujeres deberia comprenderse entre
esos dos valores para aceptar H0, de lo contrario, por descarte seria la
segunda hipotesis la ganadora.

\begin{Shaded}
\begin{Highlighting}[]
\KeywordTok{mean}\NormalTok{(data_female}\OperatorTok{$}\NormalTok{ratio)}
\end{Highlighting}
\end{Shaded}

\begin{verbatim}
## [1] 0.04713949
\end{verbatim}

no se puede aceptar H0, por lo que las mujeres definitivamente tienen
una tasa de suicidio muy inferior a la de los hombres.

\subsubsection{\texorpdfstring{\textbf{\emph{PUNTO
2}}}{PUNTO 2}}\label{punto-2}

primeramente se van a dividir en las dos categorias

\begin{Shaded}
\begin{Highlighting}[]
\NormalTok{data_agregated <-}\StringTok{ }\KeywordTok{aggregate}\NormalTok{(}\KeywordTok{cbind}\NormalTok{(data_fixed_v2}\OperatorTok{$}\NormalTok{population, data_fixed_v2}\OperatorTok{$}\NormalTok{suicides_no, data_fixed_v2}\OperatorTok{$}\NormalTok{ratio), }\DataTypeTok{by=}\KeywordTok{list}\NormalTok{(}\DataTypeTok{Category=}\NormalTok{data_fixed_v2}\OperatorTok{$}\NormalTok{country), }\DataTypeTok{FUN=}\NormalTok{sum)}
\NormalTok{mediana<-}\KeywordTok{median}\NormalTok{(data_agregated}\OperatorTok{$}\NormalTok{V1)}
\NormalTok{data_big_countries<-data_agregated[data_agregated}\OperatorTok{$}\NormalTok{V1 }\OperatorTok{>=}\StringTok{ }\NormalTok{mediana, ]}
\NormalTok{data_small_countries<-data_agregated[data_agregated}\OperatorTok{$}\NormalTok{V1 }\OperatorTok{<}\StringTok{ }\NormalTok{mediana, ]}
\NormalTok{data_big_ordered <-}\StringTok{ }\NormalTok{data_big_countries[}\KeywordTok{order}\NormalTok{(data_big_countries[}\DecValTok{2}\NormalTok{]),]}
\NormalTok{data_small_ordered <-}\StringTok{ }\NormalTok{data_small_countries[}\KeywordTok{order}\NormalTok{(data_small_countries[}\DecValTok{2}\NormalTok{]),]}
\end{Highlighting}
\end{Shaded}

\begin{verbatim}
Paises con menor poblacion
\end{verbatim}

\begin{Shaded}
\begin{Highlighting}[]
\KeywordTok{head}\NormalTok{(}\KeywordTok{cbind}\NormalTok{(data_small_ordered}\OperatorTok{$}\NormalTok{Category, data_small_ordered}\OperatorTok{$}\NormalTok{V1),}\DataTypeTok{n=}\DecValTok{10}\NormalTok{)}
\end{Highlighting}
\end{Shaded}

\begin{verbatim}
##       [,1]                    [,2]    
##  [1,] "Cayman Islands"        "28400" 
##  [2,] "Bermuda"               "98500" 
##  [3,] "Saint Kitts and Nevis" "112800"
##  [4,] "San Marino"            "163440"
##  [5,] "Sao Tome and Principe" "260100"
##  [6,] "Macau"                 "336814"
##  [7,] "Dominica"              "374200"
##  [8,] "Rodrigues"             "408345"
##  [9,] "Cabo Verde"            "439643"
## [10,] "Kiribati"              "732927"
\end{verbatim}

\begin{verbatim}
Paises con mayor poblacion
\end{verbatim}

\begin{Shaded}
\begin{Highlighting}[]
\KeywordTok{tail}\NormalTok{(}\KeywordTok{cbind}\NormalTok{(data_big_ordered}\OperatorTok{$}\NormalTok{Category, data_big_ordered}\OperatorTok{$}\NormalTok{V1), }\DataTypeTok{n=}\DecValTok{10}\NormalTok{)}
\end{Highlighting}
\end{Shaded}

\begin{verbatim}
##       [,1]                       [,2]        
## [50,] "Thailand"                 "1801002288"
## [51,] "France"                   "1816722978"
## [52,] "Germany"                  "1856678452"
## [53,] "Italy"                    "1873477997"
## [54,] "United Kingdom"           "1898631249"
## [55,] "Mexico"                   "3066771042"
## [56,] "Japan"                    "4014646102"
## [57,] "Russian Federation"       "4384117710"
## [58,] "Brazil"                   "5395577849"
## [59,] "United States of America" "8771678750"
\end{verbatim}

**Aclaracion: hay que tener en cuenta que la poblacion es la suma de
varios años distintos, por eso los valores a la derecha no corresponden
con la poblacion de los paises a dia de hoy, si solo se utilizase un año
seria correcto.

\textbf{H0: los paises de mayor poblacion tiene una mayor tasa de
suicidios, H1: las tasas son iguales o superior la de los paises
menores}

\begin{Shaded}
\begin{Highlighting}[]
\CommentTok{#97% ---> alfa=1-0.97 --> alfa= 0.03}
\CommentTok{#P(Z<z) = 1-alfa/2 = 1-0.03/2 = 0.985}
\CommentTok{#Según las tablas P(Z<z)=0.985 --> z=2.17}
\NormalTok{var1 <-}\StringTok{ }\NormalTok{data_big_ordered}\OperatorTok{$}\NormalTok{V3}
\NormalTok{##por un lado:}
\NormalTok{extremosuperior <-}\StringTok{ }\KeywordTok{mean}\NormalTok{(var1) }\OperatorTok{+}\StringTok{ }\FloatTok{2.17} \OperatorTok{*}\StringTok{ }\KeywordTok{sd}\NormalTok{(var1)}\OperatorTok{/}\KeywordTok{sqrt}\NormalTok{(}\KeywordTok{length}\NormalTok{(var1))}
\NormalTok{extremoinferior <-}\StringTok{ }\KeywordTok{mean}\NormalTok{(var1) }\OperatorTok{-}\StringTok{ }\FloatTok{2.17} \OperatorTok{*}\StringTok{ }\KeywordTok{sd}\NormalTok{(var1)}\OperatorTok{/}\KeywordTok{sqrt}\NormalTok{(}\KeywordTok{length}\NormalTok{(var1))}
\KeywordTok{cat}\NormalTok{(}\StringTok{"rango de aceptacion: ["}\NormalTok{, extremoinferior,}\StringTok{","}\NormalTok{,extremosuperior,}\StringTok{"]"}\NormalTok{)}
\end{Highlighting}
\end{Shaded}

\begin{verbatim}
## rango de aceptacion: [ 29.25453 , 42.82235 ]
\end{verbatim}

\begin{Shaded}
\begin{Highlighting}[]
\KeywordTok{mean}\NormalTok{( data_small_ordered}\OperatorTok{$}\NormalTok{V3)}
\end{Highlighting}
\end{Shaded}

\begin{verbatim}
## [1] 19.0645
\end{verbatim}

Por lo tanto se tiene que aceptar que el 50\% de paises con mayor
poblacion cometen unas tasas de suicidios superiores al otro 50\% con
una tasa de confianza del 97\%

\subsubsection{\texorpdfstring{5. \emph{Resolución del problema. A
partir de los resultados obtenidos, ¿cuáles son las conclusiones? ¿Los
resultados permiten responder al
problema?}}{5. Resolución del problema. A partir de los resultados obtenidos, ¿cuáles son las conclusiones? ¿Los resultados permiten responder al problema?}}\label{resolucion-del-problema.-a-partir-de-los-resultados-obtenidos-cuales-son-las-conclusiones-los-resultados-permiten-responder-al-problema}

En ambos casos ha sido posible comprobar que las hipotesis iniciales son
ciertas, los hombres cometen suicidio con una diferencia abismal en
comparacion con el subconjunto de mujeres, por otro lado, tambien ha
sido posible comprobar que los paises, a cuanto mayor poblacion, mayor
tasa de suicidios por cada 100 habitantes (este estudio se ha hecho
sobre el ratio de suicidios por poblacion del subconjunto poblacional)


\end{document}
